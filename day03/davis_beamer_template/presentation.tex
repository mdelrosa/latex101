% presentation.tex

% Editing Author: Mason del Rosario 
% Description: A modified version of Pavillet's Oxonion template for UC Davis

% Copyright 2019 Clara Eleonore Pavillet

% Original Author: Clara Eleonore Pavillet
% Description: This is an unofficial Oxford University Beamer Template I made from scratch. Feel free to use it, modify it, share it.
% Version: 1.0

\documentclass{beamer}
\setbeamertemplate{caption}[numbered]
\usepackage{import} % for some reason, this doesn't work when called in sty file
\input{Theme/Packages.tex}
\usepackage{bibentry}
\usepackage[utf8]{inputenc}
\usepackage[T1]{fontenc}
\usepackage{footer} % footer.sty for nofoot, blfootnote commands
\usepackage{lipsum}

\usetheme{oxonian}
\nobibliography* % enables bibentry command in blfootnote commands

\title {Davis Beamer Template}
\titlegraphic{\includegraphics[width=3cm]{Theme/Logos/DavisLogoV1.png}}
\subtitle{\LaTeX{} 101}
\author{Jane Doe}
\institute{UC Davis GradPathways}
\date{\today}

\begin{document}

    \begin{frame}
      % Print the title page as the first slide
      \titlepage
    \end{frame}

    \label{Outline}
    \begin{frame}{Outline}
      \tableofcontents
    \end{frame}

    \section{General \LaTeX{} Environments}

    \subsection{Unordered Lists}

    \begin{frame}{Unordered Lists}{Itemize environment (default)}
    \begin{itemize}
        \item First item
        \item Second item
        \item Third item
    \end{itemize}
    \end{frame}

    \begin{frame}{Ordered Lists}{Itemize environment (different shapes)}
    \setbeamertemplate{itemize items}[circle]
    \begin{itemize}
        \item First item
        \item Second item
        \item Third item
    \end{itemize}

    \setbeamertemplate{itemize items}[square]
    \begin{itemize}
        \item First item
        \item Second item
        \item Third item
    \end{itemize}

    \setbeamertemplate{itemize items}[ball]
    \begin{itemize}
        \item First item
        \item Second item
        \item Third item
    \end{itemize}

    \setbeamertemplate{itemize items}[default] % reset

    \end{frame}

    \subsection{Ordered Lists}

    \begin{frame}{Ordered Lists}{Enumerate environment (default)}
    \begin{enumerate}
        \item First item
        \item Second item
        \item Third item
    \end{enumerate}
    \end{frame}

    \begin{frame}{Ordered Lists}{Enumerate environment (different shapes)}
    \setbeamertemplate{enumerate items}[circle]
    \begin{enumerate}
        \item First item
        \item Second item
        \item Third item
    \end{enumerate}

    \setbeamertemplate{enumerate items}[square]
    \begin{enumerate}
        \item First item
        \item Second item
        \item Third item
    \end{enumerate}

    \setbeamertemplate{enumerate items}[ball]
    \begin{enumerate}
        \item First item
        \item Second item
        \item Third item
    \end{enumerate}

    \setbeamertemplate{enumerate items}[default] % reset

    \end{frame}

    \subsection{Descriptions}

    \begin{frame}{Descriptions}{Description environment}

    % item commands take optional argument for decorating each item
    \begin{description}
        \item[Day 01] Basics
        \item[Day 02] Advanced Topics
        \item[Day 03] Templates
    \end{description}

    \end{frame}

    \subsection{Tables}

    \begin{frame}{Tables}

    \begin{table}
    \begin{tabular}{| c || c | c |}
        \hline
        No. & Name & Age \\
        \hline \hline
        1 & John T & 24 \\
        2 & Norman P & 8 \\
        3 & Alex K & 14 \\ 
        \hline
    \end{tabular}
    \caption{Name and age of students}
    \end{table}

    \end{frame}

    \subsection{Figures}

    % Figures in beamer
    \begin{frame}{Figures}
    \begin{figure}
        \includegraphics[width=0.8\linewidth]{Theme/Logos/DavisLogoV3.png}
        \caption{A logo for UC Davis}
    \end{figure}
    \end{frame}

    \subsection{Footnotes}

    \nofoot{
    \begin{frame}{Footnotes}
        Text with a footnote\footnote{Here is the footnote.}.
    \end{frame}
    }

    \subsection{Citations}

    % blfootnote = make space for in-frame references to complement citations
    %              wrapper for bibentry command; use with frame surrounded by nofoot command
    \nofoot{
    \begin{frame}{Citations}
      An example of a citation \cite{ref:oetiker1995not} and a footnote reference.
      \blfootnote{\bibentry{ref:oetiker1995not}}
    \end{frame}
    }

    \section*{References}

    % references slide
    \nofoot{
    \begin{frame}{References}
      \setbeamertemplate{bibliography item}[text]
      \bibliographystyle{ieeetr}
      \bibliography{refs}
    \end{frame}
    }

    \section{Beamer-specific Environments}

    \subsection{Columns}

    \begin{frame}{Columns}{\texttt{columns}, \texttt{column} environment}
    \begin{columns}
        \begin{column}{0.5\textwidth} % first column
            \tiny{\lipsum[1-1]}
        \end{column}

        \begin{column}{0.5\textwidth} % second column
            \includegraphics[width=0.9\columnwidth]{Theme/Logos/DavisLogoV3.png}
        \end{column}
    \end{columns}
    \end{frame}

    \subsection{Blocks}

    \label{Blocks}
    \begin{frame}{Blocks}{Generic blocks}

    \begin{block}{Block 1}
    This is a simple block in beamer.
    \end{block}

    \begin{alertblock}{Block 2}
    This is an alert block in beamer.
    \end{alertblock}

    \begin{exampleblock}{Block 3}
    This is an example block in beamer.
    \end{exampleblock}

    \end{frame}


    \begin{frame}{Blocks}{Math blocks}

    \begin{theorem}
        It's in \LaTeX{} so it must be true $ a^2 + b^2 = c^2$.
    \end{theorem}
    \begin{corollary}
        a = b
    \end{corollary}
    \begin{proof}
        a + b = b + c
    \end{proof}
    \end{frame}
    \subsection{Hyperlinks and Buttons}
    \begin{frame}{Hyperlinks and Buttons}

        \hyperlink{Blocks}{Click here to return to `Blocks'}

        \hyperlink{Outline}{\beamerbutton{Click here to return to `Outline'}}
    \end{frame}

\end{document}

% \begin{document}

%   \footnotesize{
%   {\setbeamertemplate{footline}{} 
%   \frame{\titlepage}}

%   \section*{Outline}

%   % Example outline frame; renders sections, subsections, etc.
%   \begin{frame}{Outline}
%     \tableofcontents
%   \end{frame}

%   \section{Introduction}

%     % Example frame for introducing a section 
%     \begin{frame}[plain]
%       \vfill
%       \centering
%       \begin{beamercolorbox}[sep=8pt,center,shadow=true,rounded=true]{}
%         \usebeamerfont{title}\insertsectionhead\par%
%         \color{davisblue}\noindent\rule{10cm}{1pt} \\
%         \footnotesize{Section subtitle}
%       \end{beamercolorbox}
%       \vfill
%     \end{frame}
    
%   \subsection{Images}

%     \begin{frame}{Example Image}
%       \begin{figure}
%         \includegraphics[width=0.7\linewidth]{example-image}
%         \caption{\texttt{example-image} is provided by default in most \LaTeX distributions.}
%         \label{fig:example}
%       \end{figure}
%     \end{frame}

%     % nofoot = suppresses the footer bar/badge.
%     %          provides more space in a given frame.
%     \nofoot{
%     \begin{frame}{Example Image (no Footer)}
%       \begin{figure}
%         \includegraphics[width=0.8\linewidth]{example-image}
%         \caption{\texttt{example-image} with footer suppressed.}
%         \label{fig:example-nofoot}
%       \end{figure}
%     \end{frame}
%     }

%   \subsection{Citations}

%     % blfootnote = make space for in-frame references to complement citations
%     %              wrapper for bibentry command; use with frame surrounded by nofoot command
%     \nofoot{
%     \begin{frame}{Example Image (no Footer)}
%       An example of a citation \cite{ref:oetiker1995not} and a footnote reference.
%       \blfootnote{\bibentry{ref:oetiker1995not}}
%     \end{frame}
%     }

%   \section*{References}

%     % references slide
%     \nofoot{
%     \begin{frame}{References}
%       \setbeamertemplate{bibliography item}[text]
%       \bibliographystyle{ieeetr}
%       \bibliography{refs}
%     \end{frame}
%     }

% \end{document}
