\documentclass[twocolumn]{article}
\usepackage[utf8]{inputenc}
\usepackage[margin=0.5in]{geometry}

\usepackage{lipsum}
\usepackage{xcolor}
\newcommand\greenline[1]{\textcolor{green}{\underline{#1}}}

\title{Day 01 Exercise}
\date{March 2022}

\begin{document}

\maketitle

\section{Introduction} \label{sec:intro}

Your goal is to reverse-engineer this document! Here's what to do:

\begin{enumerate}
    \item Open Overleaf, sign in.
    \item Create a blank project by:
    \begin{enumerate}
        \item Clicking on ``New Project" 
        \item Clicking on ``Blank Project"
    \end{enumerate}
    \item Fill in the project so it looks like this page!
\end{enumerate}

\section{Define a command} \label{sec:command}

Using the \texttt{kw} command as a template, write a command called \texttt{greenline} to format text \greenline{like this}.

\section{Use the \texttt{lipsum} package} \label{sec:package}

Using the \texttt{lipsum} command from the \texttt{lipsum} package, generate the following three paragraphs of dummy text.

\lipsum[1-3]

\section{Hints}

Unless you want more practice, you don't have to replicate this section!

\begin{itemize}
    \item \textbf{Formatting}
    \begin{itemize}
        \item The numbered headers use \texttt{section} commands.
        \item The margins have been adjusted to 0.5 in.
        \item The \texttt{[twocolumn]} option for \texttt{documentclass} creates two columns.
        \item Text formatting we've used includes \texttt{texttt}.
    \end{itemize}
    \item \textbf{\ref{sec:intro} Introduction}
    \begin{itemize}
        \item There is a nested \texttt{enumerate} environment here!
    \end{itemize}
    \item \textbf{\ref{sec:command} Define a command}
    \begin{itemize}
        \item You might need to use the \texttt{xcolor} package to get the color green!
        \item The syntax for defining a command is: \texttt{\textbackslash newcommand\textbackslash greenline[]\{\}}
    \end{itemize}
    \item \textbf{\ref{sec:package} Use the \texttt{lipsum} package}
    \begin{itemize}
        \item You'll need to call the \texttt{\textbackslash lipsum} command with an optional argument.
    \end{itemize}
\end{itemize}


\end{document}