% day01.tex

% Copyright 2019 Clara Eleonore Pavillet

% Author: Clara Eleonore Pavillet
% Description: This is an unofficial Oxford University Beamer Template I made from scratch. Feel free to use it, modify it, share it.
% Version: 1.0

\documentclass{beamer}
\usepackage{import} % for some reason, this doesn't work when called in sty file
\input{Theme/Packages.tex}
\usepackage{lecture_notes}
\usepackage{bibentry}
\usepackage[utf8]{inputenc}
\usepackage[T1]{fontenc}

\graphicspath{ {../images/} }

\nobibliography*
% \usepackage[perpage]{footmisc}
\usetheme{oxonian}


\title{Day 01: Intro to \LaTeX }
\titlegraphic{\includegraphics[width=3cm]{Theme/Logos/DavisLogoV1.png}}
\author{\small{Mason del Rosario}}
\institute{\LaTeX 101}
\date{September 2021} %\today

\begin{document}
% \bibliographystyle{ieeetr}
% \nobibliography*{refs}


{\setbeamertemplate{footline}{} 
\frame{\titlepage}}

\section*{Outline}\begin{frame}{Outline}\tableofcontents\end{frame}

\section{Introduction}

  % Introduction section frame 
  \begin{frame}[plain]
    \vfill
    \centering
    \begin{beamercolorbox}[sep=8pt,center,shadow=true,rounded=true]{Background}
      \usebeamerfont{title}\insertsectionhead\par%
      \color{davisblue}\noindent\rule{10cm}{1pt} \\
      \footnotesize{Course Background}
    \end{beamercolorbox}
    \vfill
  \end{frame}
  
\subsection{Course Background}

  \begin{frame}{Course Background}
    \begin{itemize} 
      \item \textbf{Inspiration} -- https://www.learnlatex.org/. 
      \item \textbf{Slides Available} -- https://github.com/mdelrosa/latex-101.
      \begin{itemize}
        \item Template based on \href{https://www.overleaf.com/latex/templates/oxpav/xnjgrxthvjhg}{Clara Pavillet's Oxford Template}
      \end{itemize}
      \item \textbf{Slack back channel}
      \begin{itemize}
        \item https://join.slack.com/share/zt-ul82okyc-SI2GftuwPx_lFyBXll9rjw
      \end{itemize}
    \end{itemize}
  \end{frame}

\subsection{What is \LaTeX?}

  \begin{frame}{What is \LaTeX?}
    \textbf{Markup Language} -- Instructions for rendering a document.
    \begin{itemize}
      \item E.g., HTML (websites), Markdown (Github repos), SVG (vector graphics)
    \end{itemize}
  \end{frame}

\end{document}
